	\chapter{Introduction}

	Cosmic rays are highly penetrating high-energy particles (and electro-magnetic radiation), which originate outside the Earth's atmosphere, so they bomb Earth from all directions. Knowing the original composition of the radiation in question is critical, because it provides valuable information not only on the nature of the sources of such radiation, but on its evolution and nuclear synthesis suffered by them.

The interest in the detection of cosmic radiation is very evident since this radiation comes from processes that take place in outer space, typically in the stars, so it is possible to obtain information from these processes from the study of their associated radiation, through the cascades they produce when they enter the atmosphere (a particular case at a global level is the Auger Project).

The study of muon radiation has an absolutely clear interest in the whole framework of particle physics, since it is useful for testing various experimental models, including the \enquote{V-A Theory} of Quantum Field Theory. Furthermore, the attenuation coefficient of the muon is interesting for the design of accelerators, detectors, etc.

The measurements discussed in this work will be performed by the students at the Laboratory of Nuclear Physics, with the equipment available there, under the tutelage of their teacher. These measurements have already been made ​​in the past, so there are results available, which are compared with those obtained here.

In regards to the purely physical part, this work aims to explain how to measure the incident cosmic radiation at ground level, without doing any study in energy (although it is possible too) but rather in intensity (number of particles per unit time). This is due to the fact that with the experimental set-up available, it is not possible to measure the number of hits as a function of the energy of the incident particles, but rather the number of coincidences averaged through all the energies. This task is performed to separate the different components of the incident flux (hard and soft), as did other experimenters. It is intended to show that the values ​​obtained match those tabulated.

Much of the project is dedicated to calibrate the instruments that will be used to make the the measurements, and to set the system into the proper conditions for them. The technique used is one in which two organic scintillation detectors,  placed horizontally, connected to photomultipliers through a light guide, and separated by some distance, are used to take measurements of the coincidences. The advantage of doing so is, that the characteristic noise of these detectors is minimized. Furthermore, solid detectors are used because they have higher densities than gas counters, and give a reasonable probability of absorption for an average size of the detector. The main disadvantage of gas detectors is its low efficiency for many types of radiation of interest in nuclear physics (for example, the range of a 1 MeV gamma ray in air is about 100 m). Our radioactive source is the cosmos, and to filter the components, different thicknesses of lead and aluminium are used.


	This document is divided into several sections. In the theoretical introduction, some basic concepts and terms will be introduced. The experimental part contains a description of the components and the experimental technique as it should ideally be done, although in the Results section the limitations and problems that the student will encounter in practice are presented, as well as the conditions under which the measurements are actually carried out. For completeness, and to give a more linear nature to this guide, four appendices with the more tedious calculations have been included at the end, which will be referenced in the text.


	\graybox{.8}{.65}{
		\bc\textcolor{gray}{\Large{\sffamily Notation used in this document:}}\ec

		\textbf{Abbreviations:}

		EM: electro-magnetic,\\
		UV: ultra-violet,\\
		$\gamma$: gamma-rays,\\
		X: X-rays,\\
		$e^-$: electron,\\
		$\pi$: pion,\\
		$\mu$: muon,\\
		$\nu$: neutrino, etc.\vspace{2ex}

		\textbf{Units:}\\
		International System:\\
		eV: electron volts,\\
		J: Joules,\\
		C: Celsius,\\
		M: mega,\\
		G: giga, etc.\vspace{2ex}

		\textbf{Chemical symbols:}\\
		The elements of the periodic table.\vspace{2ex}

		\textbf{References:}\\
		There are internal (marked in \textcolor{red}{red}) and external (marked in \textcolor{blue}{blue}) references.\vspace{2ex}
	}
